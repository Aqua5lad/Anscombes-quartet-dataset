
% Default to the notebook output style

    


% Inherit from the specified cell style.




    
\documentclass[11pt]{article}

    
    
    \usepackage[T1]{fontenc}
    % Nicer default font (+ math font) than Computer Modern for most use cases
    \usepackage{mathpazo}

    % Basic figure setup, for now with no caption control since it's done
    % automatically by Pandoc (which extracts ![](path) syntax from Markdown).
    \usepackage{graphicx}
    % We will generate all images so they have a width \maxwidth. This means
    % that they will get their normal width if they fit onto the page, but
    % are scaled down if they would overflow the margins.
    \makeatletter
    \def\maxwidth{\ifdim\Gin@nat@width>\linewidth\linewidth
    \else\Gin@nat@width\fi}
    \makeatother
    \let\Oldincludegraphics\includegraphics
    % Set max figure width to be 80% of text width, for now hardcoded.
    \renewcommand{\includegraphics}[1]{\Oldincludegraphics[width=.8\maxwidth]{#1}}
    % Ensure that by default, figures have no caption (until we provide a
    % proper Figure object with a Caption API and a way to capture that
    % in the conversion process - todo).
    \usepackage{caption}
    \DeclareCaptionLabelFormat{nolabel}{}
    \captionsetup{labelformat=nolabel}

    \usepackage{adjustbox} % Used to constrain images to a maximum size 
    \usepackage{xcolor} % Allow colors to be defined
    \usepackage{enumerate} % Needed for markdown enumerations to work
    \usepackage{geometry} % Used to adjust the document margins
    \usepackage{amsmath} % Equations
    \usepackage{amssymb} % Equations
    \usepackage{textcomp} % defines textquotesingle
    % Hack from http://tex.stackexchange.com/a/47451/13684:
    \AtBeginDocument{%
        \def\PYZsq{\textquotesingle}% Upright quotes in Pygmentized code
    }
    \usepackage{upquote} % Upright quotes for verbatim code
    \usepackage{eurosym} % defines \euro
    \usepackage[mathletters]{ucs} % Extended unicode (utf-8) support
    \usepackage[utf8x]{inputenc} % Allow utf-8 characters in the tex document
    \usepackage{fancyvrb} % verbatim replacement that allows latex
    \usepackage{grffile} % extends the file name processing of package graphics 
                         % to support a larger range 
    % The hyperref package gives us a pdf with properly built
    % internal navigation ('pdf bookmarks' for the table of contents,
    % internal cross-reference links, web links for URLs, etc.)
    \usepackage{hyperref}
    \usepackage{longtable} % longtable support required by pandoc >1.10
    \usepackage{booktabs}  % table support for pandoc > 1.12.2
    \usepackage[inline]{enumitem} % IRkernel/repr support (it uses the enumerate* environment)
    \usepackage[normalem]{ulem} % ulem is needed to support strikethroughs (\sout)
                                % normalem makes italics be italics, not underlines
    

    
    
    % Colors for the hyperref package
    \definecolor{urlcolor}{rgb}{0,.145,.698}
    \definecolor{linkcolor}{rgb}{.71,0.21,0.01}
    \definecolor{citecolor}{rgb}{.12,.54,.11}

    % ANSI colors
    \definecolor{ansi-black}{HTML}{3E424D}
    \definecolor{ansi-black-intense}{HTML}{282C36}
    \definecolor{ansi-red}{HTML}{E75C58}
    \definecolor{ansi-red-intense}{HTML}{B22B31}
    \definecolor{ansi-green}{HTML}{00A250}
    \definecolor{ansi-green-intense}{HTML}{007427}
    \definecolor{ansi-yellow}{HTML}{DDB62B}
    \definecolor{ansi-yellow-intense}{HTML}{B27D12}
    \definecolor{ansi-blue}{HTML}{208FFB}
    \definecolor{ansi-blue-intense}{HTML}{0065CA}
    \definecolor{ansi-magenta}{HTML}{D160C4}
    \definecolor{ansi-magenta-intense}{HTML}{A03196}
    \definecolor{ansi-cyan}{HTML}{60C6C8}
    \definecolor{ansi-cyan-intense}{HTML}{258F8F}
    \definecolor{ansi-white}{HTML}{C5C1B4}
    \definecolor{ansi-white-intense}{HTML}{A1A6B2}

    % commands and environments needed by pandoc snippets
    % extracted from the output of `pandoc -s`
    \providecommand{\tightlist}{%
      \setlength{\itemsep}{0pt}\setlength{\parskip}{0pt}}
    \DefineVerbatimEnvironment{Highlighting}{Verbatim}{commandchars=\\\{\}}
    % Add ',fontsize=\small' for more characters per line
    \newenvironment{Shaded}{}{}
    \newcommand{\KeywordTok}[1]{\textcolor[rgb]{0.00,0.44,0.13}{\textbf{{#1}}}}
    \newcommand{\DataTypeTok}[1]{\textcolor[rgb]{0.56,0.13,0.00}{{#1}}}
    \newcommand{\DecValTok}[1]{\textcolor[rgb]{0.25,0.63,0.44}{{#1}}}
    \newcommand{\BaseNTok}[1]{\textcolor[rgb]{0.25,0.63,0.44}{{#1}}}
    \newcommand{\FloatTok}[1]{\textcolor[rgb]{0.25,0.63,0.44}{{#1}}}
    \newcommand{\CharTok}[1]{\textcolor[rgb]{0.25,0.44,0.63}{{#1}}}
    \newcommand{\StringTok}[1]{\textcolor[rgb]{0.25,0.44,0.63}{{#1}}}
    \newcommand{\CommentTok}[1]{\textcolor[rgb]{0.38,0.63,0.69}{\textit{{#1}}}}
    \newcommand{\OtherTok}[1]{\textcolor[rgb]{0.00,0.44,0.13}{{#1}}}
    \newcommand{\AlertTok}[1]{\textcolor[rgb]{1.00,0.00,0.00}{\textbf{{#1}}}}
    \newcommand{\FunctionTok}[1]{\textcolor[rgb]{0.02,0.16,0.49}{{#1}}}
    \newcommand{\RegionMarkerTok}[1]{{#1}}
    \newcommand{\ErrorTok}[1]{\textcolor[rgb]{1.00,0.00,0.00}{\textbf{{#1}}}}
    \newcommand{\NormalTok}[1]{{#1}}
    
    % Additional commands for more recent versions of Pandoc
    \newcommand{\ConstantTok}[1]{\textcolor[rgb]{0.53,0.00,0.00}{{#1}}}
    \newcommand{\SpecialCharTok}[1]{\textcolor[rgb]{0.25,0.44,0.63}{{#1}}}
    \newcommand{\VerbatimStringTok}[1]{\textcolor[rgb]{0.25,0.44,0.63}{{#1}}}
    \newcommand{\SpecialStringTok}[1]{\textcolor[rgb]{0.73,0.40,0.53}{{#1}}}
    \newcommand{\ImportTok}[1]{{#1}}
    \newcommand{\DocumentationTok}[1]{\textcolor[rgb]{0.73,0.13,0.13}{\textit{{#1}}}}
    \newcommand{\AnnotationTok}[1]{\textcolor[rgb]{0.38,0.63,0.69}{\textbf{\textit{{#1}}}}}
    \newcommand{\CommentVarTok}[1]{\textcolor[rgb]{0.38,0.63,0.69}{\textbf{\textit{{#1}}}}}
    \newcommand{\VariableTok}[1]{\textcolor[rgb]{0.10,0.09,0.49}{{#1}}}
    \newcommand{\ControlFlowTok}[1]{\textcolor[rgb]{0.00,0.44,0.13}{\textbf{{#1}}}}
    \newcommand{\OperatorTok}[1]{\textcolor[rgb]{0.40,0.40,0.40}{{#1}}}
    \newcommand{\BuiltInTok}[1]{{#1}}
    \newcommand{\ExtensionTok}[1]{{#1}}
    \newcommand{\PreprocessorTok}[1]{\textcolor[rgb]{0.74,0.48,0.00}{{#1}}}
    \newcommand{\AttributeTok}[1]{\textcolor[rgb]{0.49,0.56,0.16}{{#1}}}
    \newcommand{\InformationTok}[1]{\textcolor[rgb]{0.38,0.63,0.69}{\textbf{\textit{{#1}}}}}
    \newcommand{\WarningTok}[1]{\textcolor[rgb]{0.38,0.63,0.69}{\textbf{\textit{{#1}}}}}
    
    
    % Define a nice break command that doesn't care if a line doesn't already
    % exist.
    \def\br{\hspace*{\fill} \\* }
    % Math Jax compatability definitions
    \def\gt{>}
    \def\lt{<}
    % Document parameters
    \title{Anscombes Dataset Analysis}
    
    
    

    % Pygments definitions
    
\makeatletter
\def\PY@reset{\let\PY@it=\relax \let\PY@bf=\relax%
    \let\PY@ul=\relax \let\PY@tc=\relax%
    \let\PY@bc=\relax \let\PY@ff=\relax}
\def\PY@tok#1{\csname PY@tok@#1\endcsname}
\def\PY@toks#1+{\ifx\relax#1\empty\else%
    \PY@tok{#1}\expandafter\PY@toks\fi}
\def\PY@do#1{\PY@bc{\PY@tc{\PY@ul{%
    \PY@it{\PY@bf{\PY@ff{#1}}}}}}}
\def\PY#1#2{\PY@reset\PY@toks#1+\relax+\PY@do{#2}}

\expandafter\def\csname PY@tok@w\endcsname{\def\PY@tc##1{\textcolor[rgb]{0.73,0.73,0.73}{##1}}}
\expandafter\def\csname PY@tok@c\endcsname{\let\PY@it=\textit\def\PY@tc##1{\textcolor[rgb]{0.25,0.50,0.50}{##1}}}
\expandafter\def\csname PY@tok@cp\endcsname{\def\PY@tc##1{\textcolor[rgb]{0.74,0.48,0.00}{##1}}}
\expandafter\def\csname PY@tok@k\endcsname{\let\PY@bf=\textbf\def\PY@tc##1{\textcolor[rgb]{0.00,0.50,0.00}{##1}}}
\expandafter\def\csname PY@tok@kp\endcsname{\def\PY@tc##1{\textcolor[rgb]{0.00,0.50,0.00}{##1}}}
\expandafter\def\csname PY@tok@kt\endcsname{\def\PY@tc##1{\textcolor[rgb]{0.69,0.00,0.25}{##1}}}
\expandafter\def\csname PY@tok@o\endcsname{\def\PY@tc##1{\textcolor[rgb]{0.40,0.40,0.40}{##1}}}
\expandafter\def\csname PY@tok@ow\endcsname{\let\PY@bf=\textbf\def\PY@tc##1{\textcolor[rgb]{0.67,0.13,1.00}{##1}}}
\expandafter\def\csname PY@tok@nb\endcsname{\def\PY@tc##1{\textcolor[rgb]{0.00,0.50,0.00}{##1}}}
\expandafter\def\csname PY@tok@nf\endcsname{\def\PY@tc##1{\textcolor[rgb]{0.00,0.00,1.00}{##1}}}
\expandafter\def\csname PY@tok@nc\endcsname{\let\PY@bf=\textbf\def\PY@tc##1{\textcolor[rgb]{0.00,0.00,1.00}{##1}}}
\expandafter\def\csname PY@tok@nn\endcsname{\let\PY@bf=\textbf\def\PY@tc##1{\textcolor[rgb]{0.00,0.00,1.00}{##1}}}
\expandafter\def\csname PY@tok@ne\endcsname{\let\PY@bf=\textbf\def\PY@tc##1{\textcolor[rgb]{0.82,0.25,0.23}{##1}}}
\expandafter\def\csname PY@tok@nv\endcsname{\def\PY@tc##1{\textcolor[rgb]{0.10,0.09,0.49}{##1}}}
\expandafter\def\csname PY@tok@no\endcsname{\def\PY@tc##1{\textcolor[rgb]{0.53,0.00,0.00}{##1}}}
\expandafter\def\csname PY@tok@nl\endcsname{\def\PY@tc##1{\textcolor[rgb]{0.63,0.63,0.00}{##1}}}
\expandafter\def\csname PY@tok@ni\endcsname{\let\PY@bf=\textbf\def\PY@tc##1{\textcolor[rgb]{0.60,0.60,0.60}{##1}}}
\expandafter\def\csname PY@tok@na\endcsname{\def\PY@tc##1{\textcolor[rgb]{0.49,0.56,0.16}{##1}}}
\expandafter\def\csname PY@tok@nt\endcsname{\let\PY@bf=\textbf\def\PY@tc##1{\textcolor[rgb]{0.00,0.50,0.00}{##1}}}
\expandafter\def\csname PY@tok@nd\endcsname{\def\PY@tc##1{\textcolor[rgb]{0.67,0.13,1.00}{##1}}}
\expandafter\def\csname PY@tok@s\endcsname{\def\PY@tc##1{\textcolor[rgb]{0.73,0.13,0.13}{##1}}}
\expandafter\def\csname PY@tok@sd\endcsname{\let\PY@it=\textit\def\PY@tc##1{\textcolor[rgb]{0.73,0.13,0.13}{##1}}}
\expandafter\def\csname PY@tok@si\endcsname{\let\PY@bf=\textbf\def\PY@tc##1{\textcolor[rgb]{0.73,0.40,0.53}{##1}}}
\expandafter\def\csname PY@tok@se\endcsname{\let\PY@bf=\textbf\def\PY@tc##1{\textcolor[rgb]{0.73,0.40,0.13}{##1}}}
\expandafter\def\csname PY@tok@sr\endcsname{\def\PY@tc##1{\textcolor[rgb]{0.73,0.40,0.53}{##1}}}
\expandafter\def\csname PY@tok@ss\endcsname{\def\PY@tc##1{\textcolor[rgb]{0.10,0.09,0.49}{##1}}}
\expandafter\def\csname PY@tok@sx\endcsname{\def\PY@tc##1{\textcolor[rgb]{0.00,0.50,0.00}{##1}}}
\expandafter\def\csname PY@tok@m\endcsname{\def\PY@tc##1{\textcolor[rgb]{0.40,0.40,0.40}{##1}}}
\expandafter\def\csname PY@tok@gh\endcsname{\let\PY@bf=\textbf\def\PY@tc##1{\textcolor[rgb]{0.00,0.00,0.50}{##1}}}
\expandafter\def\csname PY@tok@gu\endcsname{\let\PY@bf=\textbf\def\PY@tc##1{\textcolor[rgb]{0.50,0.00,0.50}{##1}}}
\expandafter\def\csname PY@tok@gd\endcsname{\def\PY@tc##1{\textcolor[rgb]{0.63,0.00,0.00}{##1}}}
\expandafter\def\csname PY@tok@gi\endcsname{\def\PY@tc##1{\textcolor[rgb]{0.00,0.63,0.00}{##1}}}
\expandafter\def\csname PY@tok@gr\endcsname{\def\PY@tc##1{\textcolor[rgb]{1.00,0.00,0.00}{##1}}}
\expandafter\def\csname PY@tok@ge\endcsname{\let\PY@it=\textit}
\expandafter\def\csname PY@tok@gs\endcsname{\let\PY@bf=\textbf}
\expandafter\def\csname PY@tok@gp\endcsname{\let\PY@bf=\textbf\def\PY@tc##1{\textcolor[rgb]{0.00,0.00,0.50}{##1}}}
\expandafter\def\csname PY@tok@go\endcsname{\def\PY@tc##1{\textcolor[rgb]{0.53,0.53,0.53}{##1}}}
\expandafter\def\csname PY@tok@gt\endcsname{\def\PY@tc##1{\textcolor[rgb]{0.00,0.27,0.87}{##1}}}
\expandafter\def\csname PY@tok@err\endcsname{\def\PY@bc##1{\setlength{\fboxsep}{0pt}\fcolorbox[rgb]{1.00,0.00,0.00}{1,1,1}{\strut ##1}}}
\expandafter\def\csname PY@tok@kc\endcsname{\let\PY@bf=\textbf\def\PY@tc##1{\textcolor[rgb]{0.00,0.50,0.00}{##1}}}
\expandafter\def\csname PY@tok@kd\endcsname{\let\PY@bf=\textbf\def\PY@tc##1{\textcolor[rgb]{0.00,0.50,0.00}{##1}}}
\expandafter\def\csname PY@tok@kn\endcsname{\let\PY@bf=\textbf\def\PY@tc##1{\textcolor[rgb]{0.00,0.50,0.00}{##1}}}
\expandafter\def\csname PY@tok@kr\endcsname{\let\PY@bf=\textbf\def\PY@tc##1{\textcolor[rgb]{0.00,0.50,0.00}{##1}}}
\expandafter\def\csname PY@tok@bp\endcsname{\def\PY@tc##1{\textcolor[rgb]{0.00,0.50,0.00}{##1}}}
\expandafter\def\csname PY@tok@fm\endcsname{\def\PY@tc##1{\textcolor[rgb]{0.00,0.00,1.00}{##1}}}
\expandafter\def\csname PY@tok@vc\endcsname{\def\PY@tc##1{\textcolor[rgb]{0.10,0.09,0.49}{##1}}}
\expandafter\def\csname PY@tok@vg\endcsname{\def\PY@tc##1{\textcolor[rgb]{0.10,0.09,0.49}{##1}}}
\expandafter\def\csname PY@tok@vi\endcsname{\def\PY@tc##1{\textcolor[rgb]{0.10,0.09,0.49}{##1}}}
\expandafter\def\csname PY@tok@vm\endcsname{\def\PY@tc##1{\textcolor[rgb]{0.10,0.09,0.49}{##1}}}
\expandafter\def\csname PY@tok@sa\endcsname{\def\PY@tc##1{\textcolor[rgb]{0.73,0.13,0.13}{##1}}}
\expandafter\def\csname PY@tok@sb\endcsname{\def\PY@tc##1{\textcolor[rgb]{0.73,0.13,0.13}{##1}}}
\expandafter\def\csname PY@tok@sc\endcsname{\def\PY@tc##1{\textcolor[rgb]{0.73,0.13,0.13}{##1}}}
\expandafter\def\csname PY@tok@dl\endcsname{\def\PY@tc##1{\textcolor[rgb]{0.73,0.13,0.13}{##1}}}
\expandafter\def\csname PY@tok@s2\endcsname{\def\PY@tc##1{\textcolor[rgb]{0.73,0.13,0.13}{##1}}}
\expandafter\def\csname PY@tok@sh\endcsname{\def\PY@tc##1{\textcolor[rgb]{0.73,0.13,0.13}{##1}}}
\expandafter\def\csname PY@tok@s1\endcsname{\def\PY@tc##1{\textcolor[rgb]{0.73,0.13,0.13}{##1}}}
\expandafter\def\csname PY@tok@mb\endcsname{\def\PY@tc##1{\textcolor[rgb]{0.40,0.40,0.40}{##1}}}
\expandafter\def\csname PY@tok@mf\endcsname{\def\PY@tc##1{\textcolor[rgb]{0.40,0.40,0.40}{##1}}}
\expandafter\def\csname PY@tok@mh\endcsname{\def\PY@tc##1{\textcolor[rgb]{0.40,0.40,0.40}{##1}}}
\expandafter\def\csname PY@tok@mi\endcsname{\def\PY@tc##1{\textcolor[rgb]{0.40,0.40,0.40}{##1}}}
\expandafter\def\csname PY@tok@il\endcsname{\def\PY@tc##1{\textcolor[rgb]{0.40,0.40,0.40}{##1}}}
\expandafter\def\csname PY@tok@mo\endcsname{\def\PY@tc##1{\textcolor[rgb]{0.40,0.40,0.40}{##1}}}
\expandafter\def\csname PY@tok@ch\endcsname{\let\PY@it=\textit\def\PY@tc##1{\textcolor[rgb]{0.25,0.50,0.50}{##1}}}
\expandafter\def\csname PY@tok@cm\endcsname{\let\PY@it=\textit\def\PY@tc##1{\textcolor[rgb]{0.25,0.50,0.50}{##1}}}
\expandafter\def\csname PY@tok@cpf\endcsname{\let\PY@it=\textit\def\PY@tc##1{\textcolor[rgb]{0.25,0.50,0.50}{##1}}}
\expandafter\def\csname PY@tok@c1\endcsname{\let\PY@it=\textit\def\PY@tc##1{\textcolor[rgb]{0.25,0.50,0.50}{##1}}}
\expandafter\def\csname PY@tok@cs\endcsname{\let\PY@it=\textit\def\PY@tc##1{\textcolor[rgb]{0.25,0.50,0.50}{##1}}}

\def\PYZbs{\char`\\}
\def\PYZus{\char`\_}
\def\PYZob{\char`\{}
\def\PYZcb{\char`\}}
\def\PYZca{\char`\^}
\def\PYZam{\char`\&}
\def\PYZlt{\char`\<}
\def\PYZgt{\char`\>}
\def\PYZsh{\char`\#}
\def\PYZpc{\char`\%}
\def\PYZdl{\char`\$}
\def\PYZhy{\char`\-}
\def\PYZsq{\char`\'}
\def\PYZdq{\char`\"}
\def\PYZti{\char`\~}
% for compatibility with earlier versions
\def\PYZat{@}
\def\PYZlb{[}
\def\PYZrb{]}
\makeatother


    % Exact colors from NB
    \definecolor{incolor}{rgb}{0.0, 0.0, 0.5}
    \definecolor{outcolor}{rgb}{0.545, 0.0, 0.0}



    
    % Prevent overflowing lines due to hard-to-break entities
    \sloppy 
    % Setup hyperref package
    \hypersetup{
      breaklinks=true,  % so long urls are correctly broken across lines
      colorlinks=true,
      urlcolor=urlcolor,
      linkcolor=linkcolor,
      citecolor=citecolor,
      }
    % Slightly bigger margins than the latex defaults
    
    \geometry{verbose,tmargin=1in,bmargin=1in,lmargin=1in,rmargin=1in}
    
    

    \begin{document}
    
    
    \maketitle
    
    

    
    \section{An analysis of the Anscombes Quartet
Dataset}\label{an-analysis-of-the-anscombes-quartet-dataset}

    \subsubsection{1. Background to the dataset -- who created it, when it
was created and any speculation on how it might have been
created.}\label{background-to-the-dataset-who-created-it-when-it-was-created-and-any-speculation-on-how-it-might-have-been-created.}

\paragraph{\texorpdfstring{Anscombe's quartet was constructed by the
statistician Francis Anscombe in 1973 to demonstrate both the importance
of graphing data before analyzing it and the effect of outliers on
statistical properties.
\href{https://en.wikipedia.org/wiki/Anscombe\%27s_quartet}{1}}{Anscombe's quartet was constructed by the statistician Francis Anscombe in 1973 to demonstrate both the importance of graphing data before analyzing it and the effect of outliers on statistical properties. 1}}\label{anscombes-quartet-was-constructed-by-the-statistician-francis-anscombe-in-1973-to-demonstrate-both-the-importance-of-graphing-data-before-analyzing-it-and-the-effect-of-outliers-on-statistical-properties.-1}

It comprises four datasets with nearly identical simple descriptive
statistics, yet they appear very different when graphed. Each dataset
consists of eleven (x,y) points. He described the article as being
intended to counter the impression among statisticians that "numerical
calculations are exact, but graphs are rough."
\href{https://en.wikipedia.org/wiki/Anscombe\%27s_quartet}{2}

Anscombes datasets are synthetic rather than observed values, insofaras
they were created by Anscombe. Quite how he created them is unknown.

    \begin{Verbatim}[commandchars=\\\{\}]
{\color{incolor}In [{\color{incolor}1}]:} \PY{k+kn}{import} \PY{n+nn}{numpy} \PY{k}{as} \PY{n+nn}{np}
        \PY{k+kn}{import} \PY{n+nn}{pandas} \PY{k}{as} \PY{n+nn}{pd}
        
        \PY{c+c1}{\PYZsh{} Load Anscombes datasets from a URL.}
        \PY{n}{df} \PY{o}{=} \PY{n}{pd}\PY{o}{.}\PY{n}{read\PYZus{}csv} \PY{p}{(}\PY{l+s+s2}{\PYZdq{}}\PY{l+s+s2}{https://vincentarelbundock.github.io/Rdatasets/csv/datasets/anscombe.csv}\PY{l+s+s2}{\PYZdq{}}\PY{p}{)}    
\end{Verbatim}


    \begin{Verbatim}[commandchars=\\\{\}]
{\color{incolor}In [{\color{incolor}2}]:} \PY{k+kn}{import} \PY{n+nn}{matplotlib}\PY{n+nn}{.}\PY{n+nn}{pyplot} \PY{k}{as} \PY{n+nn}{plt}
\end{Verbatim}


    \subsubsection{Task 2 (of 4) - Plot the interesting aspects of the
dataset}\label{task-2-of-4---plot-the-interesting-aspects-of-the-dataset}

    \begin{Verbatim}[commandchars=\\\{\}]
{\color{incolor}In [{\color{incolor}3}]:} \PY{n}{df}
\end{Verbatim}


\begin{Verbatim}[commandchars=\\\{\}]
{\color{outcolor}Out[{\color{outcolor}3}]:}     Unnamed: 0  x1  x2  x3  x4     y1    y2     y3     y4
        0            1  10  10  10   8   8.04  9.14   7.46   6.58
        1            2   8   8   8   8   6.95  8.14   6.77   5.76
        2            3  13  13  13   8   7.58  8.74  12.74   7.71
        3            4   9   9   9   8   8.81  8.77   7.11   8.84
        4            5  11  11  11   8   8.33  9.26   7.81   8.47
        5            6  14  14  14   8   9.96  8.10   8.84   7.04
        6            7   6   6   6   8   7.24  6.13   6.08   5.25
        7            8   4   4   4  19   4.26  3.10   5.39  12.50
        8            9  12  12  12   8  10.84  9.13   8.15   5.56
        9           10   7   7   7   8   4.82  7.26   6.42   7.91
        10          11   5   5   5   8   5.68  4.74   5.73   6.89
\end{Verbatim}
            
    \subparagraph{that's an array of all four datasets. lets pick one and
plot
it.}\label{thats-an-array-of-all-four-datasets.-lets-pick-one-and-plot-it.}

    \begin{Verbatim}[commandchars=\\\{\}]
{\color{incolor}In [{\color{incolor}10}]:} \PY{n}{e} \PY{o}{=} \PY{n}{df}\PY{p}{[}\PY{p}{[}\PY{l+s+s1}{\PYZsq{}}\PY{l+s+s1}{x1}\PY{l+s+s1}{\PYZsq{}}\PY{p}{,} \PY{l+s+s1}{\PYZsq{}}\PY{l+s+s1}{y1}\PY{l+s+s1}{\PYZsq{}}\PY{p}{]}\PY{p}{]}
         \PY{n}{e}
\end{Verbatim}


\begin{Verbatim}[commandchars=\\\{\}]
{\color{outcolor}Out[{\color{outcolor}10}]:}     x1     y1
         0   10   8.04
         1    8   6.95
         2   13   7.58
         3    9   8.81
         4   11   8.33
         5   14   9.96
         6    6   7.24
         7    4   4.26
         8   12  10.84
         9    7   4.82
         10   5   5.68
\end{Verbatim}
            
    \begin{Verbatim}[commandchars=\\\{\}]
{\color{incolor}In [{\color{incolor}11}]:} \PY{n}{plt}\PY{o}{.}\PY{n}{hist}\PY{p}{(}\PY{n}{e}\PY{p}{)}
\end{Verbatim}


\begin{Verbatim}[commandchars=\\\{\}]
{\color{outcolor}Out[{\color{outcolor}11}]:} ([array([0., 0., 0., 0., 1., 0., 1., 0., 0., 0.]),
           array([0., 0., 1., 0., 1., 0., 0., 0., 0., 0.]),
           array([0., 0., 0., 1., 0., 0., 0., 0., 0., 1.]),
           array([0., 0., 0., 0., 1., 1., 0., 0., 0., 0.]),
           array([0., 0., 0., 0., 1., 0., 0., 1., 0., 0.]),
           array([0., 0., 0., 0., 0., 1., 0., 0., 0., 1.]),
           array([0., 0., 1., 1., 0., 0., 0., 0., 0., 0.]),
           array([2., 0., 0., 0., 0., 0., 0., 0., 0., 0.]),
           array([0., 0., 0., 0., 0., 0., 1., 0., 1., 0.]),
           array([1., 0., 0., 1., 0., 0., 0., 0., 0., 0.]),
           array([0., 2., 0., 0., 0., 0., 0., 0., 0., 0.])],
          array([ 4.,  5.,  6.,  7.,  8.,  9., 10., 11., 12., 13., 14.]),
          <a list of 11 Lists of Patches objects>)
\end{Verbatim}
            
    \begin{center}
    \adjustimage{max size={0.9\linewidth}{0.9\paperheight}}{output_8_1.png}
    \end{center}
    { \hspace*{\fill} \\}
    
    \subsubsection{Task 3 (of 4) - Calculate the descriptive statistics of
the variables in the
dataset}\label{task-3-of-4---calculate-the-descriptive-statistics-of-the-variables-in-the-dataset}

    \subsubsection{Task 4 (of 4) - why the dataset is interesting, referring
to the plots and statistics
above}\label{task-4-of-4---why-the-dataset-is-interesting-referring-to-the-plots-and-statistics-above}

What's interesting about the Anscombe dataset is that while it consists
of four datasets that have identical summary statistics (e.g., mean,
standard deviation, and correlation), they produce quite dissimilar data
graphics (scatterplots).

    \section{Simple Linear Regression with
NumPy}\label{simple-linear-regression-with-numpy}

    In school, students are taught to draw lines like the following : y = 2
x + 1

They're taught to pick two values for x and calculate the corresponding
values for y using the equation. Then they draw a set of axes, plot the
points, and then draw a line extending through the two dots on their
axes.

    \begin{Verbatim}[commandchars=\\\{\}]
{\color{incolor}In [{\color{incolor}1}]:} \PY{k+kn}{import} \PY{n+nn}{matplotlib}\PY{n+nn}{.}\PY{n+nn}{pyplot} \PY{k}{as} \PY{n+nn}{plt}
        
        \PY{c+c1}{\PYZsh{} Draw some axes.}
        \PY{n}{plt}\PY{o}{.}\PY{n}{plot}\PY{p}{(}\PY{p}{[}\PY{o}{\PYZhy{}}\PY{l+m+mi}{1}\PY{p}{,} \PY{l+m+mi}{10}\PY{p}{]}\PY{p}{,} \PY{p}{[}\PY{l+m+mi}{0}\PY{p}{,} \PY{l+m+mi}{0}\PY{p}{]}\PY{p}{,} \PY{l+s+s1}{\PYZsq{}}\PY{l+s+s1}{k\PYZhy{}}\PY{l+s+s1}{\PYZsq{}}\PY{p}{)}
        \PY{n}{plt}\PY{o}{.}\PY{n}{plot}\PY{p}{(}\PY{p}{[}\PY{l+m+mi}{0}\PY{p}{,} \PY{l+m+mi}{0}\PY{p}{]}\PY{p}{,} \PY{p}{[}\PY{o}{\PYZhy{}}\PY{l+m+mi}{1}\PY{p}{,} \PY{l+m+mi}{10}\PY{p}{]}\PY{p}{,} \PY{l+s+s1}{\PYZsq{}}\PY{l+s+s1}{k\PYZhy{}}\PY{l+s+s1}{\PYZsq{}}\PY{p}{)}
        
        \PY{c+c1}{\PYZsh{} Plot the red, blue and green lines.}
        \PY{n}{plt}\PY{o}{.}\PY{n}{plot}\PY{p}{(}\PY{p}{[}\PY{l+m+mi}{1}\PY{p}{,} \PY{l+m+mi}{1}\PY{p}{]}\PY{p}{,} \PY{p}{[}\PY{o}{\PYZhy{}}\PY{l+m+mi}{1}\PY{p}{,} \PY{l+m+mi}{3}\PY{p}{]}\PY{p}{,} \PY{l+s+s1}{\PYZsq{}}\PY{l+s+s1}{b:}\PY{l+s+s1}{\PYZsq{}}\PY{p}{)}
        \PY{n}{plt}\PY{o}{.}\PY{n}{plot}\PY{p}{(}\PY{p}{[}\PY{o}{\PYZhy{}}\PY{l+m+mi}{1}\PY{p}{,} \PY{l+m+mi}{1}\PY{p}{]}\PY{p}{,} \PY{p}{[}\PY{l+m+mi}{3}\PY{p}{,} \PY{l+m+mi}{3}\PY{p}{]}\PY{p}{,} \PY{l+s+s1}{\PYZsq{}}\PY{l+s+s1}{r:}\PY{l+s+s1}{\PYZsq{}}\PY{p}{)}
         
        \PY{c+c1}{\PYZsh{} Plot the two points (1,3) and (2,5).}
        \PY{n}{plt}\PY{o}{.}\PY{n}{plot}\PY{p}{(}\PY{p}{[}\PY{l+m+mi}{1}\PY{p}{,} \PY{l+m+mi}{2}\PY{p}{]}\PY{p}{,} \PY{p}{[}\PY{l+m+mi}{3}\PY{p}{,} \PY{l+m+mi}{5}\PY{p}{]}\PY{p}{,} \PY{l+s+s1}{\PYZsq{}}\PY{l+s+s1}{ko}\PY{l+s+s1}{\PYZsq{}}\PY{p}{)}
        \PY{c+c1}{\PYZsh{} Join them with an (extending) green lines.}
        \PY{n}{plt}\PY{o}{.}\PY{n}{plot}\PY{p}{(}\PY{p}{[}\PY{o}{\PYZhy{}}\PY{l+m+mi}{1}\PY{p}{,} \PY{l+m+mi}{10}\PY{p}{]}\PY{p}{,} \PY{p}{[}\PY{o}{\PYZhy{}}\PY{l+m+mi}{1}\PY{p}{,} \PY{l+m+mi}{21}\PY{p}{]}\PY{p}{,} \PY{l+s+s1}{\PYZsq{}}\PY{l+s+s1}{g\PYZhy{}}\PY{l+s+s1}{\PYZsq{}}\PY{p}{)}
            
        \PY{c+c1}{\PYZsh{} Set some reasonable plot limits.}
        \PY{n}{plt}\PY{o}{.}\PY{n}{xlim}\PY{p}{(}\PY{p}{[}\PY{o}{\PYZhy{}}\PY{l+m+mi}{1}\PY{p}{,} \PY{l+m+mi}{10}\PY{p}{]}\PY{p}{)}
        \PY{n}{plt}\PY{o}{.}\PY{n}{ylim}\PY{p}{(}\PY{p}{[}\PY{o}{\PYZhy{}}\PY{l+m+mi}{1}\PY{p}{,} \PY{l+m+mi}{10}\PY{p}{]}\PY{p}{)}
        
        \PY{c+c1}{\PYZsh{} Show the plot.}
        \PY{n}{plt}\PY{o}{.}\PY{n}{show}\PY{p}{(}\PY{p}{)}
\end{Verbatim}


    
    \begin{verbatim}
<Figure size 640x480 with 1 Axes>
    \end{verbatim}

    
    But Simple linear regression is about the opposite problem - what if you
have some points and are looking for the equation? It's easy when the
points are perfectly on a line already, but usually real-world data has
some noise. The data might still look roughly linear, but aren't exactly
so.

    \subsection{Example (contrived and
simulated)}\label{example-contrived-and-simulated}

    Suppose you are trying to weigh your suitcase to avoid an airline's
extra charges. You don't have a weighing scales, but you do have a
spring and some gym-style weights of masses 7KG, 14KG and 21KG.

You attach the spring to the wall hook, and mark where the bottom of it
hangs. You then hang the 7KG weight on the end and mark where the bottom
of the spring is. You repeat this with the 14KG weight and the 21KG
weight. Finally, you place your case hanging on the spring, and the
spring hangs down halfway between the 7KG mark and the 14KG mark.

Is your case over the 10KG limit set by the airline?"

\paragraph{Hypothesis}\label{hypothesis}

When you look at the marks on the wall, it seems that the 0KG, 7KG, 14KG
and 21KG marks are evenly spaced. You wonder if that means your case
weighs 10.5KG.

That is, you wonder if there is a \emph{linear} relationship between the
distance the spring's hook is from its resting position, and the mass on
the end of it.

\paragraph{Experiment}\label{experiment}

You decide to experiment. You buy some new weights - a 1KG, a 2KG, a
3Kg, all the way up to 20KG. You place them each in turn on the spring
and measure the distance the spring moves from the resting position. You
tabulate the data and plot them.

\paragraph{Analysis}\label{analysis}

Here we'll import the Python libraries we need for or investigations
below.

    \begin{Verbatim}[commandchars=\\\{\}]
{\color{incolor}In [{\color{incolor}2}]:} \PY{c+c1}{\PYZsh{} Make matplotlib show interactive plots in the notebook.}
        \PY{o}{\PYZpc{}}\PY{k}{matplotlib} inline
\end{Verbatim}


    \begin{Verbatim}[commandchars=\\\{\}]
{\color{incolor}In [{\color{incolor}3}]:} \PY{c+c1}{\PYZsh{} numpy efficiently deals with numerical multi\PYZhy{}dimensional arrays.\PYZbs{}n\PYZdq{},}
        \PY{k+kn}{import} \PY{n+nn}{numpy} \PY{k}{as} \PY{n+nn}{np}
        
        \PY{c+c1}{\PYZsh{} matplotlib is a plotting library, and pyplot is its easy\PYZhy{}to\PYZhy{}use module.\PYZbs{}n\PYZdq{},}
        \PY{k+kn}{import} \PY{n+nn}{matplotlib}\PY{n+nn}{.}\PY{n+nn}{pyplot} \PY{k}{as} \PY{n+nn}{plt}
        
        \PY{c+c1}{\PYZsh{} This just sets the default plot size to be bigger.}
        \PY{n}{plt}\PY{o}{.}\PY{n}{rcParams}\PY{p}{[}\PY{l+s+s1}{\PYZsq{}}\PY{l+s+s1}{figure.figsize}\PY{l+s+s1}{\PYZsq{}}\PY{p}{]} \PY{o}{=} \PY{p}{(}\PY{l+m+mi}{8}\PY{p}{,} \PY{l+m+mi}{6}\PY{p}{)}
\end{Verbatim}


    Ignore the next couple of lines where I fake up some data. I'll use the
fact that I faked the data to explain some results later. Just pretend
that w is an array containing the weight values and d are the
corresponding distance measurements.

    \begin{Verbatim}[commandchars=\\\{\}]
{\color{incolor}In [{\color{incolor}4}]:} \PY{n}{w} \PY{o}{=} \PY{n}{np}\PY{o}{.}\PY{n}{arange}\PY{p}{(}\PY{l+m+mf}{0.0}\PY{p}{,} \PY{l+m+mf}{21.0}\PY{p}{,} \PY{l+m+mf}{1.0}\PY{p}{)}
        \PY{n}{d} \PY{o}{=} \PY{l+m+mf}{5.0} \PY{o}{*} \PY{n}{w} \PY{o}{+} \PY{l+m+mf}{10.0} \PY{o}{+} \PY{n}{np}\PY{o}{.}\PY{n}{random}\PY{o}{.}\PY{n}{normal}\PY{p}{(}\PY{l+m+mf}{0.0}\PY{p}{,} \PY{l+m+mf}{5.0}\PY{p}{,} \PY{n}{w}\PY{o}{.}\PY{n}{size}\PY{p}{)}
\end{Verbatim}


    \begin{Verbatim}[commandchars=\\\{\}]
{\color{incolor}In [{\color{incolor}5}]:} \PY{c+c1}{\PYZsh{}lets have a look at w}
        \PY{n}{w}
\end{Verbatim}


\begin{Verbatim}[commandchars=\\\{\}]
{\color{outcolor}Out[{\color{outcolor}5}]:} array([ 0.,  1.,  2.,  3.,  4.,  5.,  6.,  7.,  8.,  9., 10., 11., 12.,
               13., 14., 15., 16., 17., 18., 19., 20.])
\end{Verbatim}
            
    \begin{Verbatim}[commandchars=\\\{\}]
{\color{incolor}In [{\color{incolor}6}]:} \PY{c+c1}{\PYZsh{}lets have a look at d}
        \PY{n}{d}
\end{Verbatim}


\begin{Verbatim}[commandchars=\\\{\}]
{\color{outcolor}Out[{\color{outcolor}6}]:} array([ -0.39217182,  12.89955716,  14.78211728,  20.21884868,
                30.14672322,  33.71566819,  41.76889684,  49.24998474,
                52.37531921,  49.70933612,  56.00422399,  52.04439717,
                74.62116353,  75.65004636,  73.60021678,  96.67865331,
                89.02564241,  87.78190907,  96.1510503 , 105.78640332,
               120.22602556])
\end{Verbatim}
            
    \paragraph{lets have a look at the data from our
experiment}\label{lets-have-a-look-at-the-data-from-our-experiment}

    \begin{Verbatim}[commandchars=\\\{\}]
{\color{incolor}In [{\color{incolor}5}]:} \PY{c+c1}{\PYZsh{}Create the plot.}
        
        \PY{n}{plt}\PY{o}{.}\PY{n}{plot}\PY{p}{(}\PY{n}{w}\PY{p}{,} \PY{n}{d}\PY{p}{,} \PY{l+s+s1}{\PYZsq{}}\PY{l+s+s1}{k.}\PY{l+s+s1}{\PYZsq{}}\PY{p}{)}
         
        \PY{c+c1}{\PYZsh{} Set some properties for the plot.\PYZbs{}n\PYZdq{},}
        \PY{n}{plt}\PY{o}{.}\PY{n}{xlabel}\PY{p}{(}\PY{l+s+s1}{\PYZsq{}}\PY{l+s+s1}{Weight (KG)}\PY{l+s+s1}{\PYZsq{}}\PY{p}{)}
        \PY{n}{plt}\PY{o}{.}\PY{n}{ylabel}\PY{p}{(}\PY{l+s+s1}{\PYZsq{}}\PY{l+s+s1}{Distance (CM)}\PY{l+s+s1}{\PYZsq{}}\PY{p}{)}
        
        \PY{c+c1}{\PYZsh{} Show the plot.}
        \PY{n}{plt}\PY{o}{.}\PY{n}{show}\PY{p}{(}\PY{p}{)}
\end{Verbatim}


    \begin{Verbatim}[commandchars=\\\{\}]

        ---------------------------------------------------------------------------

        NameError                                 Traceback (most recent call last)

        <ipython-input-5-fa9e04150d0c> in <module>()
          1 \#Create the plot.
          2 
    ----> 3 plt.plot(w, d, 'k.')
          4 
          5 \# Set some properties for the plot.\textbackslash{}n",


        NameError: name 'w' is not defined

    \end{Verbatim}

    \paragraph{Model}\label{model}

It looks like the data might indeed be linear. The points don't exactly
fit on a straight line, but they are not far off it. We might put that
down to some other factors, such as the air density, or errors, such as
in our tape measure. Then we can go ahead and see what would be the best
line to fit the data.

    \paragraph{Straight lines}\label{straight-lines}

All straight lines can be expressed in the form y = mx + c. The number m
is the slope of the line. The slope is how much y increases by when x is
increased by 1.0.

The number c is the y-intercept of the line. It's the value of \(y\)
when \(x\) is 0.

    \paragraph{Fitting the model}\label{fitting-the-model}

To fit a straight line to the data, we just must pick values for m and
c.These are called the parameters of the model,and we want to pick the
best values possible for the parameters. That is, the best parameter
values \emph{given} the data observed. Below we show various lines
plotted over the data, with different values for m and c.

    \begin{Verbatim}[commandchars=\\\{\}]
{\color{incolor}In [{\color{incolor}8}]:} \PY{c+c1}{\PYZsh{} Plot w versus d with black dots.}
        \PY{n}{plt}\PY{o}{.}\PY{n}{plot}\PY{p}{(}\PY{n}{w}\PY{p}{,} \PY{n}{d}\PY{p}{,} \PY{l+s+s1}{\PYZsq{}}\PY{l+s+s1}{k.}\PY{l+s+s1}{\PYZsq{}}\PY{p}{,} \PY{n}{label}\PY{o}{=}\PY{l+s+s2}{\PYZdq{}}\PY{l+s+s2}{Data}\PY{l+s+s2}{\PYZdq{}}\PY{p}{)}
        \PY{c+c1}{\PYZsh{} Overlay some lines on the plot.}
        \PY{n}{x} \PY{o}{=} \PY{n}{np}\PY{o}{.}\PY{n}{arange}\PY{p}{(}\PY{l+m+mf}{0.0}\PY{p}{,} \PY{l+m+mf}{21.0}\PY{p}{,} \PY{l+m+mf}{1.0}\PY{p}{)}
        \PY{n}{plt}\PY{o}{.}\PY{n}{plot}\PY{p}{(}\PY{n}{x}\PY{p}{,} \PY{l+m+mf}{5.0} \PY{o}{*} \PY{n}{x} \PY{o}{+} \PY{l+m+mf}{10.0}\PY{p}{,} \PY{l+s+s1}{\PYZsq{}}\PY{l+s+s1}{r\PYZhy{}}\PY{l+s+s1}{\PYZsq{}}\PY{p}{,} \PY{n}{label}\PY{o}{=}\PY{l+s+sa}{r}\PY{l+s+s2}{\PYZdq{}}\PY{l+s+s2}{\PYZdl{}5x + 10\PYZdl{}}\PY{l+s+s2}{\PYZdq{}}\PY{p}{)}
        \PY{n}{plt}\PY{o}{.}\PY{n}{plot}\PY{p}{(}\PY{n}{x}\PY{p}{,} \PY{l+m+mf}{6.0} \PY{o}{*} \PY{n}{x} \PY{o}{+}  \PY{l+m+mf}{5.0}\PY{p}{,} \PY{l+s+s1}{\PYZsq{}}\PY{l+s+s1}{g\PYZhy{}}\PY{l+s+s1}{\PYZsq{}}\PY{p}{,} \PY{n}{label}\PY{o}{=}\PY{l+s+sa}{r}\PY{l+s+s2}{\PYZdq{}}\PY{l+s+s2}{\PYZdl{}6x +  5\PYZdl{}}\PY{l+s+s2}{\PYZdq{}}\PY{p}{)}
        \PY{n}{plt}\PY{o}{.}\PY{n}{plot}\PY{p}{(}\PY{n}{x}\PY{p}{,} \PY{l+m+mf}{5.0} \PY{o}{*} \PY{n}{x} \PY{o}{+} \PY{l+m+mf}{15.0}\PY{p}{,} \PY{l+s+s1}{\PYZsq{}}\PY{l+s+s1}{b\PYZhy{}}\PY{l+s+s1}{\PYZsq{}}\PY{p}{,} \PY{n}{label}\PY{o}{=}\PY{l+s+sa}{r}\PY{l+s+s2}{\PYZdq{}}\PY{l+s+s2}{\PYZdl{}5x + 15\PYZdl{}}\PY{l+s+s2}{\PYZdq{}}\PY{p}{)}
            
        \PY{c+c1}{\PYZsh{} Add a legend.}
        \PY{n}{plt}\PY{o}{.}\PY{n}{legend}\PY{p}{(}\PY{p}{)}
            
        \PY{c+c1}{\PYZsh{} Add axis labels.}
        \PY{n}{plt}\PY{o}{.}\PY{n}{xlabel}\PY{p}{(}\PY{l+s+s1}{\PYZsq{}}\PY{l+s+s1}{Weight (KG)}\PY{l+s+s1}{\PYZsq{}}\PY{p}{)}
        \PY{n}{plt}\PY{o}{.}\PY{n}{ylabel}\PY{p}{(}\PY{l+s+s1}{\PYZsq{}}\PY{l+s+s1}{Distance (CM)}\PY{l+s+s1}{\PYZsq{}}\PY{p}{)}
        
        \PY{c+c1}{\PYZsh{} Show the plot.}
        \PY{n}{plt}\PY{o}{.}\PY{n}{show}\PY{p}{(}\PY{p}{)}
\end{Verbatim}


    \begin{center}
    \adjustimage{max size={0.9\linewidth}{0.9\paperheight}}{output_28_0.png}
    \end{center}
    { \hspace*{\fill} \\}
    
    \subsubsection{Calculating the cost}\label{calculating-the-cost}

You can see that each of these lines roughly fits the data. Which one is
best, and is there another line that is better than all three? Is there
a "best" line?

It depends how you define the word best. Luckily, everyone seems to have
settled on what the best means. The best line is the one that minimises
the following calculated value:

\[ sum_i (y_i - mx_i - c)^2 \]

Here \((x_i, y_i)\) is the \(i^{th}\) point in the data set and sum\_i
means to sum over all points. The values of \(m\) and \(c\) are to be
determined. We usually denote the above as Cost(m, c).

Where does the above calculation come from? It's easy to explain the
part in the brackets \((y_i - mx_i - c)\). The corresponding value to
\(x_i\) in the dataset is \(y_i\). These are the measured values.

The value \(m x_i + c\) is what the model says \(y_i\) should have been.
The difference between the value that was observed (\(y_i\)) and the
value that the model gives (\(m x_i + c\)), is \(y_i - mx_i - c\).

Why square that value? Well note that the value could be positive or
negative, and you sum over all of these values. If we allow the values
to be positive or negative, then the positive could cancel the
negatives. So, the natural thing to do is to take the absolute value
\(mid y_i - m x_i - c mid\).

Well it turns out that absolute values are a pain to deal with, and
instead it was decided to just square the quantity instead, as the
square of a number is always positive. There are pros and cons to using
the square instead of the absolute value, but the square is used.

This is usually called \emph{least squares} fitting.

    \begin{Verbatim}[commandchars=\\\{\}]
{\color{incolor}In [{\color{incolor}9}]:} \PY{c+c1}{\PYZsh{} Calculate the cost of the lines above for the data above.}
        \PY{n}{cost} \PY{o}{=} \PY{k}{lambda} \PY{n}{m}\PY{p}{,}\PY{n}{c}\PY{p}{:} \PY{n}{np}\PY{o}{.}\PY{n}{sum}\PY{p}{(}\PY{p}{[}\PY{p}{(}\PY{n}{d}\PY{p}{[}\PY{n}{i}\PY{p}{]} \PY{o}{\PYZhy{}} \PY{n}{m} \PY{o}{*} \PY{n}{w}\PY{p}{[}\PY{n}{i}\PY{p}{]} \PY{o}{\PYZhy{}} \PY{n}{c}\PY{p}{)}\PY{o}{*}\PY{o}{*}\PY{l+m+mi}{2} \PY{k}{for} \PY{n}{i} \PY{o+ow}{in} \PY{n+nb}{range}\PY{p}{(}\PY{n}{w}\PY{o}{.}\PY{n}{size}\PY{p}{)}\PY{p}{]}\PY{p}{)}
        \PY{n+nb}{print}\PY{p}{(}\PY{l+s+s2}{\PYZdq{}}\PY{l+s+s2}{Cost with m = }\PY{l+s+si}{\PYZpc{}5.2f}\PY{l+s+s2}{ and c = }\PY{l+s+si}{\PYZpc{}5.2f}\PY{l+s+s2}{: }\PY{l+s+si}{\PYZpc{}8.2f}\PY{l+s+s2}{\PYZdq{}} \PY{o}{\PYZpc{}} \PY{p}{(}\PY{l+m+mf}{5.0}\PY{p}{,} \PY{l+m+mf}{10.0}\PY{p}{,} \PY{n}{cost}\PY{p}{(}\PY{l+m+mf}{5.0}\PY{p}{,} \PY{l+m+mf}{10.0}\PY{p}{)}\PY{p}{)}\PY{p}{)}
        \PY{n+nb}{print}\PY{p}{(}\PY{l+s+s2}{\PYZdq{}}\PY{l+s+s2}{Cost with m = }\PY{l+s+si}{\PYZpc{}5.2f}\PY{l+s+s2}{ and c = }\PY{l+s+si}{\PYZpc{}5.2f}\PY{l+s+s2}{: }\PY{l+s+si}{\PYZpc{}8.2f}\PY{l+s+s2}{\PYZdq{}} \PY{o}{\PYZpc{}} \PY{p}{(}\PY{l+m+mf}{6.0}\PY{p}{,}  \PY{l+m+mf}{5.0}\PY{p}{,} \PY{n}{cost}\PY{p}{(}\PY{l+m+mf}{6.0}\PY{p}{,}  \PY{l+m+mf}{5.0}\PY{p}{)}\PY{p}{)}\PY{p}{)}
        \PY{n+nb}{print}\PY{p}{(}\PY{l+s+s2}{\PYZdq{}}\PY{l+s+s2}{Cost with m = }\PY{l+s+si}{\PYZpc{}5.2f}\PY{l+s+s2}{ and c = }\PY{l+s+si}{\PYZpc{}5.2f}\PY{l+s+s2}{: }\PY{l+s+si}{\PYZpc{}8.2f}\PY{l+s+s2}{\PYZdq{}} \PY{o}{\PYZpc{}} \PY{p}{(}\PY{l+m+mf}{5.0}\PY{p}{,} \PY{l+m+mf}{15.0}\PY{p}{,} \PY{n}{cost}\PY{p}{(}\PY{l+m+mf}{5.0}\PY{p}{,} \PY{l+m+mf}{15.0}\PY{p}{)}\PY{p}{)}\PY{p}{)}
\end{Verbatim}


    \begin{Verbatim}[commandchars=\\\{\}]
Cost with m =  5.00 and c = 10.00:   774.98
Cost with m =  6.00 and c =  5.00:  1874.55
Cost with m =  5.00 and c = 15.00:  1579.54

    \end{Verbatim}

    \paragraph{Minimising the cost}\label{minimising-the-cost}

We want to calculate values of \(m\) and \(c\) that give the lowest
value for the cost value above. For our given data set we can plot the
cost value/function. Recall that the cost is:

\[ Cost(m, c) = \\sum_i (y_i - mx_i - c)^2 \]

This is a function of two variables, \(m\) and \(c\), so a plot of it is
three dimensional. See the \textbf{Advanced} section below for the plot.

In the case of fitting a two-dimensional line to a few data points, we
can easily calculate exactly the best values of \(m\) and \(c\). Some of
the details are discussed in the \textbf{Advanced} section, as they
involve calculus, but the resulting code is straight-forward.

We first calculate the mean (average) values of our \(x\) values and
that of our \(y\) values. Then we subtract the mean of \(x\) from each
of the \(x\) values, and the mean of \(y\) from each of the \(y\)
values. Then we take the \emph{dot product} of the new \(x\) values and
the new \(y\) values and divide it by the dot product of the new \(x\)
values with themselves. That gives us \(m\), and we use \(m\) to
calculate \(c\).

Remember that in our dataset \(x\) is called \(w\) (for weight) and
\(y\) is called \(d\) (for distance). We calculate \(m\) and \(c\)
below.

    \begin{Verbatim}[commandchars=\\\{\}]
{\color{incolor}In [{\color{incolor}10}]:} \PY{c+c1}{\PYZsh{} Calculate the best values for m and c.}
         \PY{c+c1}{\PYZsh{}First calculate the means (a.k.a. averages) of w and d.}
         \PY{n}{w\PYZus{}avg} \PY{o}{=} \PY{n}{np}\PY{o}{.}\PY{n}{mean}\PY{p}{(}\PY{n}{w}\PY{p}{)}
         \PY{n}{d\PYZus{}avg} \PY{o}{=} \PY{n}{np}\PY{o}{.}\PY{n}{mean}\PY{p}{(}\PY{n}{d}\PY{p}{)}
         
         \PY{c+c1}{\PYZsh{} Subtract means from w and d.}
         \PY{n}{w\PYZus{}zero} \PY{o}{=} \PY{n}{w} \PY{o}{\PYZhy{}} \PY{n}{w\PYZus{}avg}
         \PY{n}{d\PYZus{}zero} \PY{o}{=} \PY{n}{d} \PY{o}{\PYZhy{}} \PY{n}{d\PYZus{}avg}
         
         \PY{c+c1}{\PYZsh{} The best m is found by the following calculation.}
         \PY{n}{m} \PY{o}{=} \PY{n}{np}\PY{o}{.}\PY{n}{sum}\PY{p}{(}\PY{n}{w\PYZus{}zero} \PY{o}{*} \PY{n}{d\PYZus{}zero}\PY{p}{)} \PY{o}{/} \PY{n}{np}\PY{o}{.}\PY{n}{sum}\PY{p}{(}\PY{n}{w\PYZus{}zero} \PY{o}{*} \PY{n}{w\PYZus{}zero}\PY{p}{)}
         \PY{c+c1}{\PYZsh{} Use m from above to calculate the best c.}
         \PY{n}{c} \PY{o}{=} \PY{n}{d\PYZus{}avg} \PY{o}{\PYZhy{}} \PY{n}{m} \PY{o}{*} \PY{n}{w\PYZus{}avg}
         \PY{n+nb}{print}\PY{p}{(}\PY{l+s+s2}{\PYZdq{}}\PY{l+s+s2}{m is }\PY{l+s+si}{\PYZpc{}8.6f}\PY{l+s+s2}{ and c is }\PY{l+s+si}{\PYZpc{}6.6f}\PY{l+s+s2}{.}\PY{l+s+s2}{\PYZdq{}} \PY{o}{\PYZpc{}} \PY{p}{(}\PY{n}{m}\PY{p}{,} \PY{n}{c}\PY{p}{)}\PY{p}{)}
\end{Verbatim}


    \begin{Verbatim}[commandchars=\\\{\}]
m is 5.308438 and c is 5.584384.

    \end{Verbatim}

    \paragraph{Best fit line}\label{best-fit-line}

So, the best values for m and c given our data and using least squares
fitting are about 4.95 for m and about 11.13 for c. We plot this line on
top of the data below.

    \begin{Verbatim}[commandchars=\\\{\}]
{\color{incolor}In [{\color{incolor}11}]:} \PY{c+c1}{\PYZsh{} Plot the best fit line.}
         \PY{n}{plt}\PY{o}{.}\PY{n}{plot}\PY{p}{(}\PY{n}{w}\PY{p}{,} \PY{n}{d}\PY{p}{,} \PY{l+s+s1}{\PYZsq{}}\PY{l+s+s1}{k.}\PY{l+s+s1}{\PYZsq{}}\PY{p}{,} \PY{n}{label}\PY{o}{=}\PY{l+s+s1}{\PYZsq{}}\PY{l+s+s1}{Original data}\PY{l+s+s1}{\PYZsq{}}\PY{p}{)}
         \PY{n}{plt}\PY{o}{.}\PY{n}{plot}\PY{p}{(}\PY{n}{w}\PY{p}{,} \PY{n}{m} \PY{o}{*} \PY{n}{w} \PY{o}{+} \PY{n}{c}\PY{p}{,} \PY{l+s+s1}{\PYZsq{}}\PY{l+s+s1}{b\PYZhy{}}\PY{l+s+s1}{\PYZsq{}}\PY{p}{,} \PY{n}{label}\PY{o}{=}\PY{l+s+s1}{\PYZsq{}}\PY{l+s+s1}{Best fit line}\PY{l+s+s1}{\PYZsq{}}\PY{p}{)}
         
         \PY{c+c1}{\PYZsh{} Add axis labels and a legend.}
         \PY{n}{plt}\PY{o}{.}\PY{n}{xlabel}\PY{p}{(}\PY{l+s+s1}{\PYZsq{}}\PY{l+s+s1}{Weight (KG)}\PY{l+s+s1}{\PYZsq{}}\PY{p}{)}
         \PY{n}{plt}\PY{o}{.}\PY{n}{ylabel}\PY{p}{(}\PY{l+s+s1}{\PYZsq{}}\PY{l+s+s1}{Distance (CM)}\PY{l+s+s1}{\PYZsq{}}\PY{p}{)}
         \PY{n}{plt}\PY{o}{.}\PY{n}{legend}\PY{p}{(}\PY{p}{)}
         
         \PY{c+c1}{\PYZsh{} Show the plot.}
         \PY{n}{plt}\PY{o}{.}\PY{n}{show}\PY{p}{(}\PY{p}{)}
\end{Verbatim}


    \begin{center}
    \adjustimage{max size={0.9\linewidth}{0.9\paperheight}}{output_34_0.png}
    \end{center}
    { \hspace*{\fill} \\}
    
    Note that the Cost of the best m and best c is not zero in this case.

    \begin{Verbatim}[commandchars=\\\{\}]
{\color{incolor}In [{\color{incolor}12}]:} \PY{n+nb}{print}\PY{p}{(}\PY{l+s+s2}{\PYZdq{}}\PY{l+s+s2}{Cost with m = }\PY{l+s+si}{\PYZpc{}5.2f}\PY{l+s+s2}{ and c = }\PY{l+s+si}{\PYZpc{}5.2f}\PY{l+s+s2}{: }\PY{l+s+si}{\PYZpc{}8.2f}\PY{l+s+s2}{\PYZdq{}} \PY{o}{\PYZpc{}} \PY{p}{(}\PY{n}{m}\PY{p}{,} \PY{n}{c}\PY{p}{,} \PY{n}{cost}\PY{p}{(}\PY{n}{m}\PY{p}{,} \PY{n}{c}\PY{p}{)}\PY{p}{)}\PY{p}{)}
\end{Verbatim}


    \begin{Verbatim}[commandchars=\\\{\}]
Cost with m =  5.31 and c =  5.58:   664.52

    \end{Verbatim}

    \subsubsection{Summary}\label{summary}

In this notebook we: 1. Investigated the data. 2. Picked a model. 3.
Picked a cost function. 4. Estimated the model parameter values that
minimised our cost function.


    % Add a bibliography block to the postdoc
    
    
    
    \end{document}
